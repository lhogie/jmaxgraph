\documentclass[11pt,a4paper]{article}
\usepackage[latin1]{inputenc}
\usepackage{amsmath}
\usepackage{amsfonts}
\usepackage{amssymb}
\title{JMaxGraph\\ dealing with huge graphs in memory}
\author{Luc Hogie}
\begin{document}
\section{Measure performance}
On the NEF Inria cluster.
\begin{itemize}
\item loading up to 2GB/s, loading the entire Twitter text ADJ file (215GB) took 7min.
\item iterating on 500M vertices takes 1s.
\end{itemize}
Loading
\section{The text ADJ file format}
First line is the number of vertices. Each following line then consists of:
\begin{enumerate}
\item the ID of the vertex;
\item the number of neighbors;
\item the ID for each neighbor.
\end{enumerate}
It takes lots of disk space: slow to read, to parse and to manipulate.
Twitter graph takes 215GB. 
On NEF, using 8 threads, reading it takes 7min.

\section{The binary ADJ file format}
The same of text ADJ, but each number/vertex is represented by a sequence of 4 bytes.
The size can be reduced up to 3.
\section{The JMG file format}
It is a directory
\begin{description}
\item[properties.txt] 
\item[label2vertex.nbs] contains a sequence of vertex ID, each coded on 4 bytes. The index of the vertex is its label. The number of vertices in the graph is the size of this file divided by 4.
\item[adj.edg] contains the adj
\item[index.nbs] contains the adj
 \end{description}
The JMG file format ensures that, after loading:
\begin{enumerate}
\item adjacency lists are ordered
\item there is no vertex in ADJlists that is not in the key set
\end{enumerate}

Takes 2-3min

\section{Computing }

\subsection{Computing reverse degrees}
17s on one thread

\subsection{Allocates}

\subsection{Building reverse ADJ}
on one thread, by allocated on the fly, 6.1min.




\end{document}